
\documentclass[11pt]{article}

% general packages without options
\usepackage{amsmath,amssymb,bbm}
% graphics
\usepackage{graphicx}
% text formatting
\usepackage[document]{ragged2e}
\usepackage{pagecolor,color}

\newcommand{\noun}[1]{\textsc{#1}}

\usepackage[utf8]{inputenc}
\usepackage[T1]{fontenc}
% geometry
\usepackage[margin=2cm]{geometry}

\usepackage{multicol}
\usepackage{setspace}

\usepackage{natbib}
\setlength{\bibsep}{0.0pt}


\usepackage{url}

%\usepackage[french]{babel}

% layout : use fancyhdr package
%\usepackage{fancyhdr}
%\pagestyle{fancy}

% variable to include comments or not in the compilation ; set to 1 to include
\def \draft {1}


% writing utilities

% comments and responses
%  -> use this comment to ask questions on what other wrote/answer questions with optional arguments (up to 4 answers)
\usepackage{xparse}
\usepackage{ifthen}
\DeclareDocumentCommand{\comment}{m o o o o}
{\ifthenelse{\draft=1}{
    \textcolor{red}{\textbf{C : }#1}
    \IfValueT{#2}{\textcolor{blue}{\textbf{A1 : }#2}}
    \IfValueT{#3}{\textcolor{ForestGreen}{\textbf{A2 : }#3}}
    \IfValueT{#4}{\textcolor{red!50!blue}{\textbf{A3 : }#4}}
    \IfValueT{#5}{\textcolor{Aquamarine}{\textbf{A4 : }#5}}
 }{}
}

% todo
\newcommand{\todo}[1]{
\ifthenelse{\draft=1}{\textcolor{red!50!blue}{\textbf{TODO : \textit{#1}}}}{}
}



\makeatletter


\makeatother


\begin{document}


\title{\vspace{-2cm}New methods and epistemologies to explore simulation models - II
\bigskip\\
\textit{Satellite Session - Conference on Complex Systems 2019}\\
\textit{Singapore - 2nd October 2019}
}

\date{}

\maketitle

\justify

\pagenumbering{gobble}


%\section*{Keywords: }\textit{Simulation model; Exploration and calibration methods}

% * Title of Satellite Session
%* Duration: 0.5/1.0/2.0 days
%* Name(s) of Organizer(s)
%* Motivation & Purpose
%* Potential Invited Speakers
%* Expected Number of Contributing Speakers & Participants


\section*{Topic}

This satellite session focuses on methods to explore, validate and calibrate simulation models, and on epistemological evolutions going along with these new practices. It aims at extending the discussions introduced at CCS2018 in the previous satellite on the exploration of simulation models. This previous session presented the OpenMOLE platform and interrogated the use of high performance computing together with optimization heuristics such as evolutionary computation for the calibration and validation of simulation models. This second session will ask similar questions but (i) aiming at a higher disciplinarity, possibly including disciplines in which simulation is not a mainstream practice either because of the difficulty to quantify such as in archeology, or because of a stronger confidence given to other methods such as analytical resolution in economics; (ii) aiming at a higher epistemological component in the discussion, to reflexively investigate how simulation and high performance computing can transform a discipline and what are the conditions for acceptably validating knowledge in that context. 


\section*{Call for papers}

Contributions are open to any research developing new methods, practices, theories and epistemologies related to models of simulation. No discipline is privileged as the debates are aimed at being interdisciplinary. Methodological contributions are as much welcomed as contributions in epistemology or history of science.

\bigskip

\noindent Abstract submission (max 500 words) is open until \textbf{30th June} on easychair at \url{https://easychair.org/conferences/?conf=simexplo2019}

% The study of complexity can not be anymore dissociated from intensive computational practices. Modeling and simulation have taken a crucial role in the extraction of knowledge, especially in the study of systems with a high complexity such as socio-technical systems. Quantitative geography is a perfect illustration of how methodological, technical, empirical and theoretical advances necessarily strongly bind together  \citep{rey2015plateforme}: the use of computation centers in the seventies would be comparable to the current democratization of grid computing which impact dramatically changes the way social science is practiced.
%This trend is propeled by the development of dedicated tools such as the OpenMOLE software for model exploration \citep{reuillon2013openmole}. It guides a progressive shift in simulation practices. Three fundamental innovative axis distinguish this new philosophy and technology compared to existing approaches in simulation: (i) the embedding of models within workflows, making model coupling and multi-modeling easier; (ii) the provision of novel heuristic methods for model exploration; and (iii) the transparent access to various intensive computation infrastructures. \cite{banos2017knowledge} emphasizes how this ``knowledge accelerator'' favors the construction of a robust and experimental social science, by the introduction of tools to deal with main requirements for it \citep{banos2013pour}: multiple heterogenous models can be compared and coupled in an interdisciplinary approach within a new incremental methodology introduced by \cite{cottineau2015modular}, models and workflows are open to ensure reproducibility, the behavior of models is better known with specific methods such as the Pattern Search Exploration algorithm \citep{10.1371/journal.pone.0138212} that provide the output feasible space of a model or the Calibration Profile algorithm \citep{reuillon2015}, multi-objective approaches to model optimization are implemented in genetic algorithms for model calibration \citep{schmitt2014half}.
%The aim of this symposium is take a reflexive positioning on these trends, situate them regarding similar practices, and establish the most crucial future issues to be tackled within that stream of research. Therefore, invited talks by \noun{D. Pumain} and \noun{R. Reuillon}, main investigators of the ERC project Geodivercity in which most of the aforementioned research took place \citep{pumain2017urban}, will first sketch the current landscape of these approaches, both from an epistemological point of view in the particular case of geography and from a methodological point of view. The point of view of these practitioner will be completed by an invited talk by \noun{F. Varenne} on the epistemology of modeling as well as a more specialized methodological invited talk by \noun{S. Carignon}. Submitted and reviewed contributions will then put these research into the broader framework of computational science and a wrap-up round-table will aim at exchanging and reflecting on future research directions.


%\section*{Format}

%\begin{itemize}
%	\item Expected full-day satellite
%	\item Expected number of contributions: 3/4 keynote speakers and 4/6 selected contributions
%	\item Expected number of participants: 20-30
%\end{itemize}



	%(possibly two days depending on the number of submissions)
% 4 invited speakers, open to submissions
% Preliminary program: general contributions from invited speakers; specific contributions from invited speakers; accepted communications



%\section*{Invited speakers}
%\noindent\noun{Dr. Elsa Arcaute}, CASA, UCL. \textit{Simulation for topological road network analysis}
%\noindent\noun{Dr. Marion Dierickx}, Harvard College Observatory, Harvard. \textit{Simulation in astrophysics for n-body problems}
%\noindent\noun{Pr. Denise Pumain}, Université Paris 1. \textit{Modeling and simulation of urban systems}





\section*{Organizers}

Juste Raimbault, Romain Reuillon (UPS CNRS 3611 Complex System Institute Paris Ile-de-France), Franck Varenne (University of Rouen, ERIAC and UMR CNRS 8590 IHPST)

%\bigskip

%\noindent\noun{Mehdi Bida}, UNIL. \textit{invited}





%%%%%%%%%%%%%%%%%%%%
%% Biblio
%%%%%%%%%%%%%%%%%%%%
%\tiny

%\begin{multicols}{2}

%\setstretch{0.3}
%\setlength{\parskip}{-0.4em}


%\bibliographystyle{apalike}
%\bibliography{biblio}
%\end{multicols}



\end{document}
