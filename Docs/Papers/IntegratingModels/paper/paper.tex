%%%%%%%%%%%%%%%%%%%%%%% file template.tex %%%%%%%%%%%%%%%%%%%%%%%%%
%
% This is a general template file for the LaTeX package SVJour3
% for Springer journals.          Springer Heidelberg 2010/09/16
%
% Copy it to a new file with a new name and use it as the basis
% for your article. Delete % signs as needed.
%
% This template includes a few options for different layouts and
% content for various journals. Please consult a previous issue of
% your journal as needed.
%
%%%%%%%%%%%%%%%%%%%%%%%%%%%%%%%%%%%%%%%%%%%%%%%%%%%%%%%%%%%%%%%%%%%
%
% First comes an example EPS file -- just ignore it and
% proceed on the \documentclass line
% your LaTeX will extract the file if required
\begin{filecontents*}{example.eps}
%!PS-Adobe-3.0 EPSF-3.0
%%BoundingBox: 19 19 221 221
%%CreationDate: Mon Sep 29 1997
%%Creator: programmed by hand (JK)
%%EndComments
gsave
newpath
  20 20 moveto
  20 220 lineto
  220 220 lineto
  220 20 lineto
closepath
2 setlinewidth
gsave
  .4 setgray fill
grestore
stroke
grestore
\end{filecontents*}
%
\RequirePackage{fix-cm}
%
%\documentclass{svjour3}                     % onecolumn (standard format)
%\documentclass[smallcondensed]{svjour3}     % onecolumn (ditto)
\documentclass[smallextended]{svjour3}       % onecolumn (second format)
%\documentclass[twocolumn]{svjour3}          % twocolumn
%
\smartqed  % flush right qed marks, e.g. at end of proof
%
\usepackage{graphicx}
%
% \usepackage{mathptmx}      % use Times fonts if available on your TeX system
%
% insert here the call for the packages your document requires
%\usepackage{latexsym}
% etc.
%
% please place your own definitions here and don't use \def but
% \newcommand{}{}
%
% Insert the name of "your journal" with
% \journalname{myjournal}
%
\begin{document}

\title{Integrating and validating urban simulation models}
\subtitle{}

%\titlerunning{Short form of title}        % if too long for running head

\author{Juste Raimbault}

%\authorrunning{Short form of author list} % if too long for running head

\institute{J. Raimbault \at
              Center for Advanced Spatial Analysis, University College London\\
              \email{j.raimbault@ucl.ac.uk}
}

%\date{Received: date / Accepted: date}
\date{}
% The correct dates will be entered by the editor


\maketitle

\begin{abstract}
\vspace{-1cm}
Urban systems are intrinsically complex, involving different dimensions and scales, and consequently various approaches and scientific disciplines. In that context, urban simulation models have been coined as essential for the construction of evidence-based and integrated urban sciences. This review and position paper synthesises previous work focused on coupling and integrating urban models on the one hand, and exploring and validating such simulation models on the other hand. These research directions are complementary basis for a research program towards the development of integrated urban theories, with some application perspectives to sustainable territorial planning.
\keywords{Urban simulation models \and Model coupling \and Model validation \and Sustainable territorial policies}
\end{abstract}


% TODO for final version (pub process:?)
% - https://www.sciencedirect.com/science/article/abs/pii/016792369390022U
% - https://www.sciencedirect.com/science/article/abs/pii/S1364815210000435
% - https://www.sciencedirect.com/science/article/abs/pii/S1364815203001506



\section{Introduction}


Cities, and more generally urban and territorial systems, are highly complex to understand, manage and plan. They imply imbricated scales, from the intra-urban microscopic scales with individual agents such as household or companies, through the intermediate mesoscopic scale of the urban area, to the macroscopic scale of systems of cities \cite{pumain2008socio}. Many different dimensions and viewpoints can be considered as an entry into urban issues, and several disciplines relate to emerging urban sciences \cite{batty2013new}, including urban economics, urban geography, planning and design. Complexity approaches to cities are therefore a relevant interdisciplinary take into urban systems \cite{batty2012origins}.

Furthermore, the question of sustainable cities and territories is crucial in terms of policy application of urban sciences, especially in the current context of global change and pressing need for sustainable transitions. Cities can be in that context as much an issue than a solution, and models with sufficient endogeneity can help to anticipate their dynamics. The integration of environmental dimensions within simulation models for territorial systems is in that context an important aspect, but the coupling between dimensions remains generally weak or to a certain level of exogeneity. For example, \cite{viguie2014downscaling} use an urban growth model to evaluate the local impact of climate change, but does not make associated aspects endogenous, such as energy price or the energy efficiency of the urban structure. Negative externalities such as congestion and pollution, directly linked to emissions, have a feedback for example on the location of activities. Similarly, numerous studies in ecology which establish the impact of anthropic habitat disturbances would particularly benefit from a coupling with urban growth models, for example for a better management of the interface between the city and nature within the new urban regimes that are urban mega-regions \cite{hall2006polycentric}. At the macroscopic scale, several couplings between models of systems of cities and ecological models or from environmental science can also be considered. For example issues related to the production, storage and distribution of energy are endogenous to territorial systems, and the dynamics of infrastructures and associated entities can be integrated within models for systems of cities \cite{pumain2018evolutionary}.


In that context, we propose a research program rooted into complexity science to foster the construction of bridges and interdisciplinary dialogues in urban science, in some sense the construction of integrated urban theories. The concept of integration can be understood through the complex systems roadmap \cite{bourgine2009french}. It combines horizontal integration (fundamental transversal questions at the intersection of different types of complex systems) with vertical integration (multiple levels coupled within multi-scale models). The integration between knowledge domains in the sense of \cite{raimbault2017applied}, (theoretical, empirical, modeling, data, methods and tools knowledge domains) is also an aspect of the expected integration, especially through the strong link between models, empirical data and methods, with their exploration and validation.

We indeed directly inherit from the evolutionary theory of cities developed by Denise Pumain \cite{pumain2017urban}, in particular regarding the role of simulation models and model validation methods - the OpenMOLE platform and associated methods \cite{reuillon2013openmole} were mainly developed in this frame. We assume thus that the construction of integrated models can not be achieved independently from their systematic exploration and validation. The project is thus built on three complementary axis: (i) the coupling of heterogenous urban models to integrate dimensions; (ii) the coupling of models at different scales for the vertical integration; and (iii) the development of new spatial simulation model validation methods, simultaneously applied to previous models.

The integration of urban models has a long history linked to the development of quantitative approaches to urban systems and operational urban models \cite{batty2021defining}. Land-use Transport Interaction models for example, which aim at simulating the impact of transportation infrastructure on different components of territories, generally include several sub-models such as transport models, cellular automatons for the evolution of land-use, or economic microsimulation models \cite{wegener2021land}. The framework proposed by \cite{batty2021new} implements spatial interaction modeling between population and employment on large-scale urban system, providing a basis for coupling with other dimensions and other types of models. More recently, the development of integrated models for smart cities has also put forward the relevance of model coupling in urban simulation. \cite{yamagata2013simulating} couple for example models for land-use and energy consumption. \cite{muvuna2019methodology} introduce a framework to integrate information from different components of the smart city.

The question of integrating models and dimensions has also been the focus of other disciplines not directly related to urban issues. Ecology has for example studied the interaction between society, ecosystems and resources \cite{bithell2009coupling}. The link between economics and ecology is a crucial issue regarding sustainability \cite{hofkes1996modelling}. Resource economics is a field in which model coupling can act as a medium for smoother interdisciplinary collaborations \cite{macleod2016model}. Within urban science, such research directions remain mostly to be explored, from diverse viewpoints including methodology and theory.


We describe each axis of our research program in the rest of this paper, mostly by developing a synthesis of previous research at the basis of this project. We finally discuss how the construction of such integrated models could be applied to the design of sustainable territorial planning policies.


\section{Horizontal integration: coupling urban models and dimensions}


\subsection{Model coupling}

An horizontal integration between complementary dimensions of urban systems can be achieved through the coupling of different urban models, taking into account different dimensions. The example of urban dynamics and environmental issues developed above can be investigated regarding several issues related to climate change, such as flooding \cite{ford2019multi} or heat waves \cite{lemonsu2015vulnerability}. The integration of economy and geography is also a question that remains to be mostly explored \cite{marchionni2004geographical}. More generally, the Sustainable Development Goals are declined into 17 goals and 169 targets which must be tackled in an integrative way \cite{stafford2017integration}.

Some previous work, building on the evolutionary theory of cities and at the inception of this project, focuses on the co-evolution of transportation networks and territories. It illustrates how such model coupling can be achieved. \cite{raimbault2019urban} couples a reaction-diffusion urban morphogenesis model with a multi-model of transportation network growth to yield a co-evolution model at the mesoscopic scale. At the scale of the system of cities, \cite{raimbault2021modeling} integrates an abstract urban network evolution model into an urban dynamics model for the population of cities. These models effectively achieve a strong coupling, as they capture circular causality between the transportation network and cities \cite{raimbault2020unveiling}. A refinement of the network growth heuristic for the macro model, described by \cite{raimbault2020hierarchy}, introduces an intermediate scale of geographical description, and is a first step towards multi-scale models.

Current work recently presented as \cite{raimbault2021estimating} and based on the methodology of \cite{raimbault2021building} focuses on coupling models to build open transport models from the bottom-up using open source components: the MATSim transport modeling framework \cite{horni2016multi} is for example integrated with the Quant spatial interaction model \cite{batty2021new} and a population microsimulation model \cite{lomax2017microsimulation}. The implementation of model coupling using the OpenMOLE workflow system \cite{passerat2017reproducible} illustrates the joint development and exploration of models. Expected policy results of this work integrate an additional dimension, through the implementation of density indicators in public transport which can be used as a proxy for potential contamination in the epidemiological context of COVID-19.



\subsection{Literature mapping}

Model coupling between disciplines is made simpler when a reflexive positioning is available, for example using scientometrics and literature mapping techniques. Therefore, \cite{raimbault2019exploration} provides a framework to jointly explore interdisciplinary citation and semantic networks, while \cite{raimbault2021empowering} develop open tools to explore scientific corpuses. 

Literature mapping and systematic reviews can also be used to identify and compare existing models with great detail, as for the case of land-use transport interactions: \cite{raimbault2021interdisciplinary} proposes a map of the interactions between disciplines involved in such modeling (from geography to urban economics, planning and physics) while \cite{raimbault2019meta} proceeds to a meta-analysis of model characteristics based on their scientific context. Such efforts can also be a basis towards more advanced model benchmarking experiments, as \cite{raimbault2020comparison} in the case of several urban morphogenesis models in terms of produced urban morphologies; these benchmark being in turn a step towards model integration or multi-modeling approaches.

\subsection{Open issues}


Several research questions remain rather open in the case of model coupling for urban systems, although some methodological and technical contributions could be imported from other disciplines to tackle them.

For instance, the definition of model coupling itself is not clear. A serial coupling of models (outputs of one becoming inputs of the other) can be qualified as weak or loose coupling \cite{clarke1998loose}, while the integration of feedback loops between model states at each time step, or the construction of a more general model including the two, can be seen as a strong coupling. Measures to quantify a degree of coupling also remain to be investigated.

Beyond the nature of the coupling, technical aspects must be investigated. When models have different time steps or even different time scales, implementations can not directly communicate. The DEVS framework implies that coupled components follow a same formalism and can be seamlessly integrated \cite{vangheluwe2000devs}. Methodological frameworks have been introduced in physics to couple asynchronous models across different time scales \cite{lockerby2015asynchronous}.


An other dimension is wether underlying epistemologies and ontologies are intrinsically compatible. In other words, the semantic contents of models may in some cases be in contradiction. Some disciplines (or even different schools of thought in the same discipline) may introduce conflicting assumptions. Making ontologies explicit, either using established ontology systems \cite{yang2019ontology}, or a less strict approach to ontology in the case of social systems \cite{livet2010ontology}, is a way to mitigate such issues.


A last difficulty worth mentioning is the computational complexity which may arise from model coupling. In the case of a strong integration in the sense of many feedback loop, this complexity may severely increase in comparison to the computational burden of models alone; notwithstanding the fact that coupling mechanically increases the number of parameters, making the integrating model difficult to explore due to the curse of dimensionality. Simplifying expectations on model outputs, working on patterns only as does the approach of Pattern Oriented Modeling \cite{grimm2005pattern}, may reduce uncertainties due to a higher complexity. Specific methods, such as inverse problems methods \cite{vogel2002computational}, allow targeted exploration of the parameter space and a reduction of complexity. One must still notice that different types of complexity are related \cite{raimbault2020relating}, and thus there is generally no straightforward way in terms of computation to take into account emergence in a model.



\section{Vertical integration: constructing multi-scale models}


The construction of multi-scale models is in itself a crucial aspect towards integrated models and theories. Following \cite{rozenblat2018conclusion}, systems of cities have reached multiple levels of articulation and interdependencies. This implies that their management and planning must necessarily be multi-scalar in order to take into account geographical particularities, while still ensuring a global consistence which is a condition for limited inequalities between territories.


Moreover, considerable methodological work is required to elaborate coupling methods between scales, as for example the hybrid modeling coupling agent-based models with differential equations for an epidemiological model \cite{banos2015coupling}. This allows determining the relevance of levels to be included for tackling a particular problem and avoid so-called ontological dead-ends (i.e. include appropriate levels of representation) \cite{roth2006reconstruction}. Such methodological investigation are also necessary to understand the nature of retroactions between scales to be included, and when these are necessary or not. A crucial aspect mostly neglected in the literature is to effectively achieve a strong coupling between scales, in the sense of including both upward and downward feedbacks.

Recent work paved the way towards such multi-scale strongly coupled models. \cite{raimbault2021strong} couples the urban dynamics model of \cite{raimbault2020indirect} at the scale of the system of cities with the urban morphogenesis model of \cite{raimbault2018calibration} at the scale of the urban area. Downward feedback is captured through policy parameters, as local development decisions generally react to the global integration of a given city, while upward feedback is taken into account by combining positive and negative externalities within the urban area to update its global performance and interaction parameters with other cities.

To understand the interplay between urban form at the microscopic scale, transportation network development and developer agents decisions at the mesoscopic scale, \cite{raimbault2021multiscale} proposes a stylised agent-based model achieving a strong coupling between the microscopic and mesoscopic scale.

These efforts remain at the stage of stylised and rather simple model, and the construction of data-driven and validated multi-scale models remains to be explored. Coupling between scales is also a way to couple between different urban dimensions, and both integration are not necessarily independent.


\section{Model exploration and validation}



\subsection{Model validation methods}

Working to integrate models and theories necessarily implies a better understanding of the modeling process itself, but also of stylised dynamics produced by simulation models, in the sense of patterns \cite{grimm2005pattern}.

In the case of geographical models for urban systems, such a knowledge has for example been developed within the \emph{Geodivercity} European Research Council project lead by Denise Pumain \cite{pumain2017urban}, which included to the conception of new model validation methods. These methods were elaborated specifically to tackle thematic geographical questions, but were applied to many contexts thereafter, witnessing a beneficial relationship both from the geography and computer science viewpoints \cite{raimbault2019methods}. More generally in social sciences, modeling and simulation driven by new practices including model coupling, the use of high-performance computing for model exploration, and open science practices, are tightly linked to the production of a new type of integrated knowledge \cite{banos2013pour}. This is in some sense an aspect of the computational shift in contemporary science coined by \cite{arthur2014complexity}.


Such model validation methods can be applied for example to assess the necessity and sufficiency of processes in a multi-modeling context, in the case of the Calibration Profile algorithm \cite{reuillon2015new}. Genetic algorithms distributed on a computation grid using OpenMOLE are a highly efficient tool for model calibration \cite{schmitt2015half}. The search for diversity in model outputs, obtained with the Pattern Space Exploration algorithm, can be used to explore the space of feasible configurations a model can produce, and possibly highlight unexpected behavior \cite{cherel2015beyond}.


Therefore, the development of specific methods and tools to improve the extraction of knowledge from simulation models, and an epistemological investigation on modeling practices, are crucial within this project. Methods for the exploration, sensitivity analysis, and validation, are essential for robust application of models, but also yield a better complementarity with other types of approaches since they can establish when modeling is not relevant anymore.


\subsection{Spatial sensitivity analysis}

Recent work has focused on the development of validation methods in the specific case of spatial simulation models. More particularly, a methodology coined as \emph{Spatial Sensitivity Analysis} has been introduced by \cite{raimbault2019space} in order to understand the effect of spatial initial conditions on model outcomes. This provides a generic way to disentangle effects which are intrinsic to model dynamics from contingent effects due to the geography. The generation of synthetic spatial configuration for territorial systems is required, and \cite{raimbault2019second} has investigated the generation of such data while controlling correlation patterns. \cite{raimbault2019generating} introduce and compare multiple generators for building configurations at the district scale. These methods are implemented into a single scala library, allowing an easier integration into the OpenMOLE software \cite{raimbault2020scala}.


\subsection{Further developments}

Different directions are worth exploring regarding the development of validation methods specific to spatial urban simulation models. Null models are rather rare in the field, in comparison to ecology for example which has used the Neutral Landscape Model for quite some time \cite{with1997use}. Bayesian calibration methods such as particle filters \cite{jiang2017new} are not widely used for urban models, although notable exceptions exists in domains close to engineering such as the study of traffic \cite{marinica2011particle}. Finally, some work on the nature and definition of validation itself would be required. Different standards are expected depending on disciplines and types of models, since for example a computational model will always be more difficult to trust than a tractable analytical model. Methods developed will depend on such definition and expected standards of robustness.



\section{Discussion: towards evidence-based multi-scalar sustainable territorial planning}


The strong complementarity of the three axis of the research program, elaborated on previous and current work described above, stems naturally from the intrinsic link between vertical and horizontal integrations, and from the integration of knowledge domains as model validation methods are elaborated jointly as models are constructed and explored. A fully open issue is the transfer of integrated models towards policy applications.

Models have multiple functions, for which a precise typology has been introduced by \cite{varenne2018theories}. Regarding the theoretical and quantitative geography \cite{cuyala2014analyse} side of our proposal, in terms of theoretical contribution, models are instruments of knowledge production and contribute to the construction of theories. Transferring models towards policy applications requires to diversify their functions, and the end users. In a perspectivist view of knowledge production \cite{giere2010scientific}, models can not be dissociated from the question they are designed to answer and from the entity formulating it. Model application implies thus an upstream design, or redesign, of models such that planners, policy makers, stakeholders, and the public - any relevant user indeed - are blent in the modelling process from the beginning.


Concrete research directions have also to be investigated for a possible application of models. A first research direction lies in the modeling of policy and governance itself. The management of sustainability implies according to \cite{etzion2018management} new governance structures. Their modeling is still not well integrated within spatially explicit models. \cite{le2015modeling} has proposed to integrate transportation network governance into a land-use transport model to yield a co-evolution model. \cite{raimbault2020coevolution} uses game-theory to simulate transportation network investments at the scale of the system of cities.


One other crucial research axis towards the application of integrated models to policies is the construction of harmonised multi-dimensional databases. Such databases must (i) consider consistent geographical entities in time, in the sense of the dynamical ontology proposed by \cite{bretagnolle2009villes}; (ii) integrate multiple dimensions to allow coupling urban dynamics with socio-economic, environmental, sustainability and global change issues; (iii) be open, accessible, documented and reproducible. Technical aspects of data integration will also have to be carefully taken into account. In that regard, contributions from digital twins modeling (a model simulating a system in real time and having potential feedback on its functioning) applied to urban systems \cite{batty2018digital}, or tools and methods to handle information elaborated in the context of smart cities \cite{shahrour2017smart}, are important contributions.


Finally, models must be constructed jointly with the precision of their application context to ensure their potential applicability. Obstacles to applicability can raise on diverse dimensions. Regarding the aforementioned model end users, many different paths can be taken to ensure a co-construction, such as companion modeling \cite{drogoul2015agent} in which local stakeholders participate to the construction and exploration of the model. Issues related to real-time model visualisation are also an important technical problem to take into account for model applications \cite{milton2019accelerating}. Scientific mediation initiatives are an other way to diffuse the results of modeling experiments into policy applications.



\section{Conclusion}

The diversity and complexity of urban systems requires a plurality of viewpoints \cite{pumain2020conclusion}, which call for integration between dimensions and scales. We have synthesised in this paper a current stream of research focusing around three axis, namely (i) horizontal integration through model coupling; (ii) vertical integration through the construction of multi-scale urban models; and (iii) development of model validation and exploration methods. These strongly complementary axis form the basis of a long term research program, with a goal of application to sustainable territorial policies.


% BibTeX users please use one of
%\bibliographystyle{unsrt}
%\bibliographystyle{spbasic}      % basic style, author-year citations
%\bibliographystyle{spmpsci}      % mathematics and physical sciences
%\bibliographystyle{spphys}       % APS-like style for physics
%\bibliography{biblio.bib}   % name your BibTeX data base

\begin{thebibliography}{10}

\bibitem{pumain2008socio}
Denise Pumain.
\newblock The socio-spatial dynamics of systems of cities and innovation
  processes: a multi-level model.
\newblock In {\em The Dynamics of Complex Urban Systems}, pages 373--389.
  Springer, 2008.

\bibitem{batty2013new}
Michael Batty.
\newblock {\em The new science of cities}.
\newblock MIT press, 2013.

\bibitem{batty2012origins}
Michael Batty and Stephen Marshall.
\newblock The origins of complexity theory in cities and planning.
\newblock In {\em Complexity theories of cities have come of age}, pages
  21--45. Springer, 2012.

\bibitem{viguie2014downscaling}
Vincent Vigui{\'e}, St{\'e}phane Hallegatte, and Julie Rozenberg.
\newblock Downscaling long term socio-economic scenarios at city scale: a case
  study on paris.
\newblock {\em Technological forecasting and social change}, 87:305--324, 2014.

\bibitem{hall2006polycentric}
Peter~Geoffrey Hall and Kathy Pain.
\newblock {\em The polycentric metropolis: learning from mega-city regions in
  Europe}.
\newblock Routledge, 2006.

\bibitem{pumain2018evolutionary}
Denise Pumain.
\newblock An evolutionary theory of urban systems.
\newblock In {\em International and transnational perspectives on urban
  systems}, pages 3--18. Springer, 2018.

\bibitem{bourgine2009french}
Paul Bourgine, David Chavalarias, Edith Perrier, Frederic Amblard, Francois
  Arlabosse, Pierre Auger, Jean-Bernard Baillon, Olivier Barreteau, Pierre
  Baudot, Elisabeth Bouchaud, et~al.
\newblock French roadmap for complex systems 2008-2009.
\newblock {\em arXiv preprint arXiv:0907.2221}, 2009.

\bibitem{raimbault2017applied}
Juste Raimbault.
\newblock An applied knowledge framework to study complex systems.
\newblock In {\em Complex Systems Design \& Management}, pages 31--45, 2017.

\bibitem{pumain2017urban}
Denise Pumain and Romain Reuillon.
\newblock {\em Urban dynamics and simulation models}.
\newblock Springer, 2017.

\bibitem{reuillon2013openmole}
Romain Reuillon, Mathieu Leclaire, and Sebastien Rey-Coyrehourcq.
\newblock Openmole, a workflow engine specifically tailored for the distributed
  exploration of simulation models.
\newblock {\em Future Generation Computer Systems}, 29(8):1981--1990, 2013.

\bibitem{batty2021defining}
Michael Batty.
\newblock Defining urban science.
\newblock {\em Urban Informatics}, 2021.

\bibitem{wegener2021land}
Michael Wegener.
\newblock Land-use transport interaction models.
\newblock {\em Handbook of regional science}, pages 229--246, 2021.

\bibitem{batty2021new}
Michael Batty and Richard Milton.
\newblock A new framework for very large-scale urban modelling.
\newblock {\em Urban Studies}, page 0042098020982252, 2021.

\bibitem{yamagata2013simulating}
Yoshiki Yamagata and Hajime Seya.
\newblock Simulating a future smart city: An integrated land use-energy model.
\newblock {\em Applied Energy}, 112:1466--1474, 2013.

\bibitem{muvuna2019methodology}
Jules Muvuna, Tuleen Boutaleb, Keith~J Baker, and Slobodan~B Mickovski.
\newblock A methodology to model integrated smart city system from the
  information perspective.
\newblock {\em Smart Cities}, 2(4):496--511, 2019.

\bibitem{bithell2009coupling}
Mike Bithell and James Brasington.
\newblock Coupling agent-based models of subsistence farming with
  individual-based forest models and dynamic models of water distribution.
\newblock {\em Environmental Modelling \& Software}, 24(2):173--190, 2009.

\bibitem{hofkes1996modelling}
Marjan~W Hofkes.
\newblock Modelling sustainable development: An economy-ecology integrated
  model.
\newblock {\em Economic Modelling}, 13(3):333--353, 1996.

\bibitem{macleod2016model}
Miles MacLeod and Michiru Nagatsu.
\newblock Model coupling in resource economics: conditions for effective
  interdisciplinary collaboration.
\newblock {\em Philosophy of science}, 83(3):412--433, 2016.

\bibitem{ford2019multi}
Alistair Ford, Stuart Barr, Richard Dawson, James Virgo, Michael Batty, and Jim
  Hall.
\newblock A multi-scale urban integrated assessment framework for climate
  change studies: A flooding application.
\newblock {\em Computers, Environment and Urban Systems}, 75:229--243, 2019.

\bibitem{lemonsu2015vulnerability}
Aude Lemonsu, Vincent Viguie, M~Daniel, and V~Masson.
\newblock Vulnerability to heat waves: Impact of urban expansion scenarios on
  urban heat island and heat stress in paris (france).
\newblock {\em Urban Climate}, 14:586--605, 2015.

\bibitem{marchionni2004geographical}
Caterina Marchionni.
\newblock Geographical economics versus economic geography: towards a
  clarification of the dispute.
\newblock {\em Environment and Planning A}, 36(10):1737--1753, 2004.

\bibitem{stafford2017integration}
Mark Stafford-Smith, David Griggs, Owen Gaffney, Farooq Ullah, Belinda Reyers,
  Norichika Kanie, Bjorn Stigson, Paul Shrivastava, Melissa Leach, and Deborah
  O’Connell.
\newblock Integration: the key to implementing the sustainable development
  goals.
\newblock {\em Sustainability science}, 12(6):911--919, 2017.

\bibitem{raimbault2019urban}
Juste Raimbault.
\newblock An urban morphogenesis model capturing interactions between networks
  and territories.
\newblock In {\em The mathematics of urban morphology}, pages 383--409.
  Springer, 2019.

\bibitem{raimbault2021modeling}
Juste Raimbault.
\newblock Modeling the co-evolution of cities and networks.
\newblock In {\em Handbook of Cities and Networks, in press}. E. Elgar.

\bibitem{raimbault2020unveiling}
Juste Raimbault.
\newblock Unveiling co-evolutionary patterns in systems of cities: a systematic
  exploration of the simpopnet model.
\newblock In {\em Theories and Models of Urbanization}, pages 261--278.
  Springer, 2020.

\bibitem{raimbault2020hierarchy}
Juste Raimbault.
\newblock Hierarchy and co-evolution processes in urban systems.
\newblock {\em arXiv preprint arXiv:2001.11989}, 2020.

\bibitem{raimbault2021estimating}
Juste Raimbault and Michael Batty.
\newblock Estimating public transport congestion in uk urban areas with open
  transport models.
\newblock {\em 29th Annual GIS Research UK Conference (GISRUK), Cardiff, Wales,
  UK (Online), 14-16 April 2021}, Apr 2021.

\bibitem{raimbault2021building}
J.~Raimbault and M.~Batty.
\newblock Building and validating modular urban transportation models using
  scientific workflow systems.
\newblock In {\em Applied Urban Modeling Global Workshop 2021}. 2021.

\bibitem{horni2016multi}
Andreas Horni, Kai Nagel, and Kay~W Axhausen.
\newblock {\em The multi-agent transport simulation MATSim}.
\newblock Ubiquity Press, 2016.

\bibitem{lomax2017microsimulation}
NM~Lomax and Andrew~P Smith.
\newblock Microsimulation for demography.
\newblock {\em Australian Population Studies}, 1(1):73--85, 2017.

\bibitem{passerat2017reproducible}
Jonathan Passerat-Palmbach, Romain Reuillon, Mathieu Leclaire, Antonios
  Makropoulos, Emma~C Robinson, Sarah Parisot, and Daniel Rueckert.
\newblock Reproducible large-scale neuroimaging studies with the openmole
  workflow management system.
\newblock {\em Frontiers in neuroinformatics}, 11:21, 2017.

\bibitem{raimbault2019exploration}
Juste Raimbault.
\newblock Exploration of an interdisciplinary scientific landscape.
\newblock {\em Scientometrics}, 119(2):617--641, 2019.

\bibitem{raimbault2021empowering}
Juste Raimbault, Pierre-Olivier Chasset, Cl{\'e}mentine Cottineau, Hadrien
  Commenges, Denise Pumain, Christine Kosmopoulos, and Arnaud Banos.
\newblock Empowering open science with reflexive and spatialised indicators.
\newblock {\em Environment and Planning B: Urban Analytics and City Science},
  48(2):298--313, 2021.

\bibitem{raimbault2021interdisciplinary}
Juste Raimbault.
\newblock An interdisciplinary bibliometric analysis of models for land-use and
  transport interactions.
\newblock {\em arXiv preprint arXiv:2102.13501}, 2021.

\bibitem{raimbault2019meta}
Juste Raimbault.
\newblock A meta-analysis of models for interactions between transportation
  networks and territories.
\newblock In {\em European Colloquium on Theoretical and Quantitative
  Geography}, 2019.

\bibitem{raimbault2020comparison}
Juste Raimbault.
\newblock A comparison of simple models for urban morphogenesis.
\newblock {\em arXiv preprint arXiv:2008.13277}, 2020.

\bibitem{clarke1998loose}
Keith~C Clarke and Leonard~J Gaydos.
\newblock Loose-coupling a cellular automaton model and gis: long-term urban
  growth prediction for san francisco and washington/baltimore.
\newblock {\em International journal of geographical information science},
  12(7):699--714, 1998.

\bibitem{vangheluwe2000devs}
Hans~LM Vangheluwe.
\newblock Devs as a common denominator for multi-formalism hybrid systems
  modelling.
\newblock In {\em Cacsd. conference proceedings. IEEE international symposium
  on computer-aided control system design (cat. no. 00th8537)}, pages 129--134.
  IEEE, 2000.

\bibitem{lockerby2015asynchronous}
Duncan~A Lockerby, Alexander Patronis, Matthew~K Borg, and Jason~M Reese.
\newblock Asynchronous coupling of hybrid models for efficient simulation of
  multiscale systems.
\newblock {\em Journal of Computational Physics}, 284:261--272, 2015.

\bibitem{yang2019ontology}
Lan Yang, Kathryn Cormican, and Ming Yu.
\newblock Ontology-based systems engineering: A state-of-the-art review.
\newblock {\em Computers in Industry}, 111:148--171, 2019.

\bibitem{livet2010ontology}
Pierre Livet, Jean-Pierre Muller, Denis Phan, and Lena Sanders.
\newblock Ontology, a mediator for agent-based modeling in social science.
\newblock {\em Journal of Artificial Societies and Social Simulation}, 13(1):3,
  2010.

\bibitem{grimm2005pattern}
Volker Grimm, Eloy Revilla, Uta Berger, Florian Jeltsch, Wolf~M Mooij, Steven~F
  Railsback, Hans-Hermann Thulke, Jacob Weiner, Thorsten Wiegand, and Donald~L
  DeAngelis.
\newblock Pattern-oriented modeling of agent-based complex systems: lessons
  from ecology.
\newblock {\em science}, 310(5750):987--991, 2005.

\bibitem{vogel2002computational}
Curtis~R Vogel.
\newblock {\em Computational methods for inverse problems}.
\newblock SIAM, 2002.

\bibitem{raimbault2020relating}
Juste Raimbault.
\newblock Relating complexities for the reflexive study of complex systems.
\newblock In {\em Theories and models of urbanization}, pages 27--41. Springer,
  2020.

\bibitem{rozenblat2018conclusion}
Celine Rozenblat and Denise Pumain.
\newblock Conclusion: Toward a methodology for multi-scalar urban system
  policies.
\newblock {\em International and Transnational Perspectives on Urban Systems},
  page 385, 2018.

\bibitem{banos2015coupling}
Arnaud Banos, Nathalie Corson, Benoit Gaudou, Vincent Laperri{\`e}re, and
  S{\'e}bastien~Rey Coyrehourcq.
\newblock Coupling micro and macro dynamics models on networks: Application to
  disease spread.
\newblock In {\em International Workshop on Multi-Agent Systems and Agent-Based
  Simulation}, pages 19--33. Springer, 2015.

\bibitem{roth2006reconstruction}
Camille Roth.
\newblock Reconstruction failures: Questioning level design.
\newblock In {\em International Workshop on Epistemological Aspects of Computer
  Simulation in the Social Sciences}, pages 89--98. Springer, 2006.

\bibitem{raimbault2021strong}
Juste Raimbault.
\newblock Strong coupling between scales in a multi-scalar model of urban
  dynamics.
\newblock {\em arXiv preprint arXiv:2101.12725}, 2021.

\bibitem{raimbault2020indirect}
Juste Raimbault.
\newblock Indirect evidence of network effects in a system of cities.
\newblock {\em Environment and Planning B: Urban Analytics and City Science},
  47(1):138--155, 2020.

\bibitem{raimbault2018calibration}
Juste Raimbault.
\newblock Calibration of a density-based model of urban morphogenesis.
\newblock {\em PloS one}, 13(9):e0203516, 2018.

\bibitem{raimbault2021multiscale}
Juste Raimbault.
\newblock A multiscale model of urban morphogenesis.
\newblock {\em arXiv preprint arXiv:2103.17241}, 2021.

\bibitem{raimbault2019methods}
Juste Raimbault and Denise Pumain.
\newblock Methods for exploring simulation models.
\newblock {\em Geographical Modeling: Cities and Territories}, 2:125--150,
  2019.

\bibitem{banos2013pour}
Arnaud Banos.
\newblock {\em Pour des pratiques de mod{\'e}lisation et de simulation
  lib{\'e}r{\'e}es en g{\'e}ographie et SHS}.
\newblock PhD thesis, Universit{\'e} Paris 1 Panth{\'e}on Sorbonne, 2013.

\bibitem{arthur2014complexity}
W~Brian Arthur.
\newblock {\em Complexity and the Economy}.
\newblock Oxford University Press, 2014.

\bibitem{reuillon2015new}
Romain Reuillon, Clara Schmitt, Ricardo De~Aldama, and Jean-Baptiste Mouret.
\newblock A new method to evaluate simulation models: the calibration profile
  (cp) algorithm.
\newblock {\em Journal of Artificial Societies and Social Simulation},
  18(1):12, 2015.

\bibitem{schmitt2015half}
Clara Schmitt, S{\'e}bastien Rey-Coyrehourcq, Romain Reuillon, and Denise
  Pumain.
\newblock Half a billion simulations: Evolutionary algorithms and distributed
  computing for calibrating the simpoplocal geographical model.
\newblock {\em Environment and Planning B: Planning and Design},
  42(2):300--315, 2015.

\bibitem{cherel2015beyond}
Guillaume Ch{\'e}rel, Cl{\'e}mentine Cottineau, and Romain Reuillon.
\newblock Beyond corroboration: Strengthening model validation by looking for
  unexpected patterns.
\newblock {\em PloS one}, 10(9):e0138212, 2015.

\bibitem{raimbault2019space}
Juste Raimbault, Cl{\'e}mentine Cottineau, Marion Le~Texier, Florent Le~Nechet,
  and Romain Reuillon.
\newblock Space matters: Extending sensitivity analysis to initial spatial
  conditions in geosimulation models.
\newblock {\em Journal of Artificial Societies and Social Simulation}, 22(4),
  2019.

\bibitem{raimbault2019second}
Juste Raimbault.
\newblock Second-order control of complex systems with correlated synthetic
  data.
\newblock {\em Complex Adaptive Systems Modeling}, 7(1):1--19, 2019.

\bibitem{raimbault2019generating}
Juste Raimbault and Julien Perret.
\newblock Generating urban morphologies at large scales.
\newblock In {\em Artificial Life Conference Proceedings}, pages 179--186. MIT
  Press, 2019.

\bibitem{raimbault2020scala}
J~Raimbault, J~Perret, and R~Reuillon.
\newblock A scala library for spatial sensitivity analysis.
\newblock GISRUK, 2020.

\bibitem{with1997use}
Kimberly~A With and Anthony~W King.
\newblock The use and misuse of neutral landscape models in ecology.
\newblock {\em Oikos}, pages 219--229, 1997.

\bibitem{jiang2017new}
Zhihong Jiang, Weigang Zhou, Hui Li, Yang Mo, Wencheng Ni, and Qiang Huang.
\newblock A new kind of accurate calibration method for robotic kinematic
  parameters based on the extended kalman and particle filter algorithm.
\newblock {\em IEEE Transactions on Industrial Electronics}, 65(4):3337--3345,
  2017.

\bibitem{marinica2011particle}
Nicolae-Emanuel Marinica and Ren{\'e} Boel.
\newblock Particle filter for platoon based models of urban traffic.
\newblock In {\em 15th WSEAS CSCC Multiconference}, pages 42--47. WSEAS Press,
  2011.

\bibitem{varenne2018theories}
Franck Varenne.
\newblock {\em Th{\'e}ories et mod{\`e}les en sciences humaines: le cas de la
  g{\'e}ographie}.
\newblock {\'E}ditions Mat{\'e}riologiques, 2018.

\bibitem{cuyala2014analyse}
Sylvain Cuyala.
\newblock {\em Analyse spatio-temporelle d'un mouvement scientifique. L'exemple
  de la g{\'e}ographie th{\'e}orique et quantitative europ{\'e}enne
  francophone.}
\newblock PhD thesis, Universit{\'e} Paris 1 Panth{\'e}on-Sorbonne, 2014.

\bibitem{giere2010scientific}
Ronald~N Giere.
\newblock {\em Scientific perspectivism}.
\newblock University of Chicago Press, 2010.

\bibitem{etzion2018management}
Dror Etzion.
\newblock Management for sustainability.
\newblock {\em Nature Sustainability}, 1(12):744--749, 2018.

\bibitem{le2015modeling}
Florent Le~N{\'e}chet and Juste Raimbault.
\newblock Modeling the emergence of metropolitan transport autorithy in a
  polycentric urban region.
\newblock In {\em European Colloqueum on Theoretical and Quantitative
  Geography}, 2015.

\bibitem{raimbault2020coevolution}
Juste Raimbault.
\newblock A co-evolution agent-based model for systems of cities and
  transportation networks integrating top-down governance through game theory.
\newblock In {\em Conference on Complex Systems 2020}, 2020.

\bibitem{bretagnolle2009villes}
Anne Bretagnolle.
\newblock {\em Villes et r{\'e}seaux de transport: des interactions dans la
  longue dur{\'e}e (France, Europe, Etats-Unis)}.
\newblock PhD thesis, 2009.

\bibitem{batty2018digital}
Michael Batty.
\newblock Digital twins, 2018.

\bibitem{shahrour2017smart}
Isam Shahrour, Lyes Alileche, and Abdurahman Alfurjani.
\newblock Smart cities: System and tools used for the digital modelling of
  physical urban systems.
\newblock In {\em 2017 Sensors Networks Smart and Emerging Technologies
  (SENSET)}, pages 1--4. IEEE, 2017.

\bibitem{drogoul2015agent}
Alexis Drogoul.
\newblock Agent-based modeling for multidisciplinary and participatory
  approaches to climate change adaptation planning.
\newblock In {\em Proceedings of RFCC-2015 Workshop, AIT}, 2015.

\bibitem{milton2019accelerating}
Richard Milton and Flora Roumpani.
\newblock Accelerating urban modelling algorithms with artificial intelligence.
\newblock In {\em Proceedings of the 5th International Conference on
  Geographical Information Systems Theory, Applications and Management},
  volume~1, pages 105--116. INSTICC, 2019.

\bibitem{pumain2020conclusion}
Denise Pumain and Juste Raimbault.
\newblock Conclusion: Perspectives on urban theories.
\newblock In {\em Theories and Models of Urbanization}, pages 303--330.
  Springer, 2020.

\end{thebibliography}



\end{document}
% end of file template.tex


%
%% For one-column wide figures use
%\begin{figure}
%% Use the relevant command to insert your figure file.
%% For example, with the graphicx package use
%  \includegraphics{example.eps}
%% figure caption is below the figure
%\caption{Please write your figure caption here}
%\label{fig:1}       % Give a unique label
%\end{figure}
%%
%% For two-column wide figures use
%\begin{figure*}
%% Use the relevant command to insert your figure file.
%% For example, with the graphicx package use
%  \includegraphics[width=0.75\textwidth]{example.eps}
%% figure caption is below the figure
%\caption{Please write your figure caption here}
%\label{fig:2}       % Give a unique label
%\end{figure*}
%%
%% For tables use
%\begin{table}
%% table caption is above the table
%\caption{Please write your table caption here}
%\label{tab:1}       % Give a unique label
%% For LaTeX tables use
%\begin{tabular}{lll}
%\hline\noalign{\smallskip}
%first & second & third  \\
%\noalign{\smallskip}\hline\noalign{\smallskip}
%number & number & number \\
%number & number & number \\
%\noalign{\smallskip}\hline
%\end{tabular}
%\end{table}

